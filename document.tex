\documentclass[a4paper, 12pt]{report}
\usepackage[utf8]{inputenc}
\usepackage{tikz}
\usepackage{graphicx}
\usepackage{geometry}
\geometry{margin=1in,left=0.6in,right=0.6in,bottom=0.5in}
\usepackage{fancyhdr}
\usepackage{hyperref}
\usepackage{chemfig}
\usetikzlibrary{calc}
\usepackage{enumitem}
\usepackage{placeins}
\usepackage{caption}
\usepackage{array}
\usepackage{float}
\usepackage{amsmath, amssymb, amscd, MnSymbol,mathrsfs}
\usepackage{tcolorbox}
\usepackage{bibref}
\usepackage{bm}
\newcommand{\vect}[1]{\boldsymbol{\mathbf{#1}}}

\usepackage{empheq}
\usepackage{pgfplots}
\pgfplotsset{compat=1.18}
\usetikzlibrary{calc, angles, quotes}
\usepackage[oldvoltagedirection]{circuitikz}
\usepackage{booktabs}
\usepackage{array}
\usepackage{arydshln}
\usepackage{tcolorbox}

\usepackage{hyperref}
\usepackage{cancel}
\usepackage{changepage}
\usepackage{placeins}
\usepackage{enumitem}
\usepackage{pgf}


\usepackage{listings}

\usepackage{minted}

\usepackage{courier}
\usepackage{xcolor}
\usepackage{color}

\lstdefinestyle{mypythonstyle}{
    language=Python,
    basicstyle=\footnotesize\ttfamily,
    numbers=left,
    numberstyle=\tiny,
    numbersep=9pt,
    showstringspaces=false,
    frame=single,
    breaklines=true,
    keywordstyle=\color{blue!80!black}\bfseries,  % Darker blue and bold for keywords
    commentstyle=\color{green!60!black},  % Softer gray for comments
    stringstyle=\color{orange},  % Different color for strings
    backgroundcolor=\color{gray!5},  % Light gray for background
    tabsize=4,
    belowcaptionskip=10pt, % Adjust space below caption
    morecomment=[l]{//},  % For inline comments
    literate={~} {$\sim$}{1},  % Converts tilde to a symbol
}

\lstdefinestyle{matlab}{
    language=MATLAB,
    basicstyle=\ttfamily\footnotesize,
    breaklines=true,
    commentstyle=\color{green!60!black},
    keywordstyle=\color{blue!80!black}\bfseries,
    stringstyle=\color{red!70!black},
    numberstyle=\tiny\color{gray},
    numbers=left,
    numbersep=9pt,
    frame=single,
    framesep=5pt,
    rulecolor=\color{black!30},
    backgroundcolor=\color{yellow!10},
    emphstyle=\color{purple}\bfseries,
    emph={[2]\%},
    emphstyle={[2]\color{green!60!black}},
    showstringspaces=false,
    tabsize=4,
    belowcaptionskip=10pt, % Adjust space below caption
    morekeywords={import,classdef,properties,methods,end,
        function,return,if,else,elseif,switch,case,otherwise,
        for,while,break,continue,try,catch,global,persistent},
    morecomment=[l][\color{magenta!70!black}]{\%\%},
    morestring=[b]',
    literate=
    *{0}{{{\color{cyan!70!black}0}}}1
    {1}{{{\color{cyan!70!black}1}}}1
    {2}{{{\color{cyan!70!black}2}}}1
    {3}{{{\color{cyan!70!black}3}}}1
    {4}{{{\color{cyan!70!black}4}}}1
    {5}{{{\color{cyan!70!black}5}}}1
    {6}{{{\color{cyan!70!black}6}}}1
    {7}{{{\color{cyan!70!black}7}}}1
    {8}{{{\color{cyan!70!black}8}}}1
    {9}{{{\color{cyan!70!black}9}}}1
}

\def\link{blue!50!black}
\makeatletter
\renewcommand*\env@matrix[1][\arraystretch]{%
    \edef\arraystretch{#1}%
    \hskip -\arraycolsep
    \let\@ifnextchar\new@ifnextchar
    \array{*{\c@MaxMatrixCols}c}
}
\makeatother


\pagestyle{fancy}
\fancyhf{}
\fancyhead[L]{Module: EG4017 - Engineering Mathematics}
\fancyhead[R]{Page \thepage}

\title{\vspace{3em} \Huge \textbf{Engineering\\ Mathematics and Computing}\\ \vspace{1em} \Large Task 1: Coursework Assessment}
\author{Student Name: Sakariye Abiikar\\ KID: 2371673}
\date{Last Updated: October 19, 2024\\ Submission Deadline: November 7, 2024, 5pm \\[1em] Git Repo : \color{blue}\url{https://github.com/sakx7/mathcompuni}}

\begin{document}
    
    \maketitle
    \thispagestyle{empty}
    
    \newpage
    \thispagestyle{empty}
    \newgeometry{margin=1in,left=0.6in,right=0.6in,bottom=0in}

    \chapter{Part A: Mathematics}
    \hrule\vspace{5em}\Large\noindent
    \textbf{Notes:}\normalsize
    
    %\begin{itemize}
    %    \item \textbf{Module Code:} EG4017
    %    \item \textbf{Module Title:} Engineering Mathematics
    %    \item \textbf{Assessment Title:} Engineering Mathematics and Computing - Task 1
    %    \item \textbf{Summative Assessment:} This assignment contributes 50\% of module grade.
    %    \item \textbf{Total Hours Required:} 25 hours per student
    %    \item \textbf{Set by:} Dr. Sergey Khaustov (RVMB122)
    %    \item \textbf{Contact:} S.Khaustov@kingston.ac.uk
    %    \item \textbf{Submission Deadline:} November 7, 2024, before 5pm
    %    \item \textbf{Feedback Release:} Formal feedback will be provided %within 20 working days after submission.
    %\end{itemize}
    
    
    \newpage\centering\restoregeometry
    
    \setcounter{page}{1}
    \fancyhead[L]{Part A: Mathematics}

    \begin{tcolorbox}[title={\color{black}{\section{Q1}}}, colback=white, colframe=black!30!white, boxrule=0.4mm, width=1\textwidth]
        Use the quotient rule to differentiate the function \( y=\frac{\ln 3 x}{2 x} \)
    \end{tcolorbox}
    
    The quotient rule states that if
    \[y = \frac{u}{v} \quad \text{then} \quad y' = \frac{u'v - uv'}{v^2}\]
    Here: \\[5pt]
    \begin{minipage}{0.4\textwidth}
        \[u = \ln 3x\] 
    \end{minipage}\hspace{-8em}
    \begin{minipage}{0.4\textwidth}
        \[v = 2x \]
    \end{minipage}\\[20pt]
    First, find the derivatives:\\[10pt]
    \begin{minipage}{0.43\textwidth}\centering
    \[u = \ln(3x)\]
    Let:
    \[\delta = 3x \qquad u = \ln(\delta)\]
    \[\frac{du}{d\delta} = \frac{1}{\delta} \qquad \frac{d\delta}{dx} = 3\]
    Using the chain rule, we can find:
    \[\frac{du}{dx} = \frac{du}{d\delta} \cdot \frac{d\delta}{dx}\]
    \[\frac{du}{dx} = \frac{3}{\delta} = \frac{3}{3x}= \frac{1}{x}\]
    General rule for composite/nested functions is solved via chain rule it's:
    \[\frac{d}{dx}(f(g(x))) = f'(g(x))\cdot g'(x)\]
    \end{minipage}\hspace{1em}\vrule\hspace{-2.5em}
    \begin{minipage}{0.4\textwidth}
        \[v = 2x\]
        \[v' = \frac{d}{dx}(2x) = 2\]
    \end{minipage}\\[30pt]
    Now we can apply the quotient rule:
    \[y' = \frac{\left(\frac{1}{x}\right)(2x) - (\ln 3x)(2)}{(2x)^2} = \frac{2 - 2\ln 3x}{4x^2}\]
    \[\boxed{y' = \frac{1 - \ln 3x}{2x^2}}\]
    
    \newpage
    \begin{tcolorbox}[title=\color{black}{\section{Q2}}, colback=white, colframe=black!30!white, boxrule=0.4mm, width=1\textwidth]
        Find the angle between the vectors \( 2 i-11 j-10 k \) and \( 5 i+8 j+7 k \)
    \end{tcolorbox}
    
    The angle \( \theta \) between two vectors \(\mathbf{a}\) and \(\mathbf{b}\) is given by:
    \[\cos \theta = \frac{\mathbf{a} \cdot \mathbf{b}}{\|\mathbf{a}\| \|\mathbf{b}\|}\]
    Calculate the dot product:
    \[\mathbf{a} \cdot \mathbf{b} = \underbrace{(2)(5)}_i + \underbrace{(-11)(8)}_j + \underbrace{(-10)(7)}_k = -148\]
    Calculate the magnitudes:
    \[\|\mathbf{a}\| = \sqrt{2^2 + (-11)^2 + (-10)^2} = \sqrt{225} = 15\]
    \[\|\mathbf{b}\| = \sqrt{5^2 + 8^2 + 7^2} = \sqrt{138}\]
    Find \(\cos \theta\):
    \[\cos \theta = \frac{-148}{15 \times \sqrt{138}}\]
    Thus, the angle \(\theta\) is:
    \[\boxed{\theta = \cos^{-1}\left(\frac{-148}{15 \times \sqrt{138}}\right) \approx 147.1^{\circ}}\]
    
    \newpage    
    \begin{tcolorbox}[title=\color{black}{\section{Q3}}, colback=white, colframe=black!30!white, boxrule=0.4mm, width=1\textwidth]
        Find the rate of change of \( y=\ln \left(16 t^{2}+19\right) \) at the specified point \( t=9 \)
    \end{tcolorbox}
        
    Differentiate \( y \) with respect to \( t \):\\[1em]
    Note prior (q1) that the general rule for composite/nested functions is:
    \[\frac{d}{dx}(f(g(x))) = f'(g(x))\cdot g'(x)\]
    Here for $y=\ln \left(16 t^{2}+19\right)$
    \[\frac{dy}{dt} = \frac{1}{16t^2 + 19} \cdot \frac{d}{dt}(16t^2 + 19)\]
    \[\frac{d}{dt}(16t^2 + 19) = 32t\]
    \[\frac{dy}{dt} = \frac{1}{16t^2 + 19} \cdot 32t = \frac{32t}{16t^2 + 19}\]
    Evaluate at \( t = 9 \):
    \[\boxed{\frac{dy}{dt} \bigg|_{t=9} = \frac{32 (9)}{16(9)^2 + 19} = \frac{288}{1315} \approx 0.219}\]
    
    \newpage
    \begin{tcolorbox}[title=\color{black}{\section{Q4}}, colback=white, colframe=black!30!white, boxrule=0.4mm, width=1\textwidth]
        Express \( \cos t-8 \sin t \) in the form \( A \cos (\omega t+\alpha) \), where \( \alpha \geq 0 \)
    \end{tcolorbox}
    you see the thing is the left side, \( \cos t - 8 \sin t \), is standard, but the right side introduces a different frequency, \( \omega \), in \( A \cos (\omega t + \alpha) \).\\[10pt]
    Since the left side has a frequency of 1, im just gonna assume \( \omega = 1 \) for simplicity. If \( \omega \) were different, you would need to rewrite the left side in terms of \( \omega t \), but this isn't specified in the problem.\\
    \vspace{0.5em}\hrule\vspace{0.5em}
    The angle subtraction formula is:
    \[A \cos(t - \phi) = A \cos(\phi) \cos(t) + A \sin(\phi) \sin(t)\]
    By comparing coefficients from both sides, we have:
    \begin{align*}
        a & = A \cos(\phi) \\
        b & = A \sin(\phi)
    \end{align*}
    To find $A$ we use $A = \sqrt{a^2 + b^2}$\\
    \vspace*{1em}
    \begin{minipage}{0.45\textwidth}\centering
        This arises from squaring both equations \(a = A \cos(\phi)\) and \(b = A \sin(\phi)\):\\[1em]
        $a^2 + b^2 = (A \cos(\phi))^2 + (A \sin(\phi))^2 = A^2 (\cos^2(\phi) + \sin^2(\phi)) = A^2$
    \end{minipage}\\
    
    \vspace{1em}
    To find the phase shift \(\phi\), we use $\tan \phi = \frac{b}{a}$\\[1em]
    \begin{minipage}{0.45\textwidth}\centering
        This comes from the definitions of sine and cosine:
        \[\tan \phi = \frac{A \sin(\phi)}{A \cos(\phi)} = \frac{b}{a}\]
    \end{minipage}\\
    \vspace{1em}
    In so we derive and make use of:
    \[A \cos(t - \phi) = a \cos(t) + b \sin(t)\]
    Where $A = \sqrt{a^2 + b^2}$ and $\tan \phi = \frac{b}{a}$\\
    \vspace{0.5em}\hrule\vspace{0.5em}
    for $b=-8$ and $a=2$
    \[A = \sqrt{1^2 + (-8)^2} = \sqrt{65}\]
    \[\tan \phi = \frac{-8}{1}=-8 \qquad \phi = \arctan(-8) \approx -82.87^\circ\]\\[6pt]
    In so plugging in gives
    \begin{align*}
    \cos t-8 \sin t &= A \cos(t - \phi)\\
    &= \sqrt{65} \cos(t - \arctan(-8))\\
    &= \sqrt{65} \cos(t - (-82.87^\circ))   
    \end{align*}
    \[\boxed{\cos t-8 \sin t \approx  8.06 \cos(t + 82.87^\circ)}\]
    
    \newpage
    \begin{tcolorbox}[title=\color{black}{\section{Q5}}, colback=white, colframe=black!30!white, boxrule=0.4mm, width=1\textwidth]
        Solve the following system of three linear equations using Cramer's rule
        \[
        \left\{\begin{array}{c}
            11 v_{1}-v_{2}+v_{3}=31.4 \\
            v_{1}+\frac{v_{2}}{2}-v_{3}=1.9 \\
            -9 v_{1}+11 v_{3}=-12
        \end{array}\right.
        \]
    \end{tcolorbox}
    
    The system can be written in matrix form \( A\mathbf{v} = \mathbf{b} \), specifically because of the variable distributions where:
    \[A = \begin{bmatrix} 11 & -1 & 1 \\ 1 & \frac{1}{2} & -1 \\ -9 & 0 & 11 \end{bmatrix}, \quad \mathbf{v} = \begin{bmatrix} v_1 \\ v_2 \\ v_3 \end{bmatrix}, \quad \mathbf{b} = \begin{bmatrix} 31.4 \\ 1.9 \\ -12 \end{bmatrix}\]
    Calculate the determinant \(\det(A)\).
    \[\det(A) = 67\]
    Now solve for each variable using Cramer's rule:
    \[A_1 = \begin{bmatrix} 31.4 & -1 & 1 \\ 1.9 & \frac{1}{2} & -1 \\ -12 & 0 & 11 \end{bmatrix}
    \quad 
    A_2 = \begin{bmatrix} 11 & 31.4 & 1 \\ 1 & 1.9 & -1 \\ -9 & -12 & 11 \end{bmatrix}
    \quad 
    A_3 = \begin{bmatrix} 11 & -1 & 31.4 \\ 1 & \frac{1}{2} & 1.9 \\ -9 & 0 & -12 \end{bmatrix}\]
    Calculate the determinants:
    \[\det(A_1)\approx187.6 \quad \det(A_2)\approx40.2\quad \det(A_3)\approx80.4\]
    \(\mathbf{v}\) can be found as:
    \[v_1 = \frac{\det(A_1)}{\det(A)}, \quad v_2 = \frac{\det(A_2)}{\det(A)}, \quad v_3 = \frac{\det(A_3)}{\det(A)}\]
    \[\boxed{\begin{array}{rcl}v_1=\frac{187.6}{67}=2.8\\[6pt]v_2=\frac{40.2}{67}=0.6\\[6pt]v_3=\frac{80.4}{67}=1.2\end{array}}\]
    There are a few ways to calculate the determinants, but especially in this working out i chose not show the method i used since it’s not the main focus of the question.\\[6pt]
    If you’re curious about how I did it, the Sarrus method for simplicity.
    \newpage    
    \begin{tcolorbox}[title=\color{black}{\section{Q6}}, colback=white, colframe=black!30!white, boxrule=0.4mm, width=1\textwidth]
        Transpose \( z = d + a \sqrt{y} \) to make \( y \) the subject.
    \end{tcolorbox}

    Starting with:
    \[z = d + a\sqrt{y}\]
    \[z - d = a\sqrt{y}\]
    \[\frac{z - d}{a} = \sqrt{y}\]
    \[\left( \frac{z - d}{a} \right)^2 = y\]
    Thus, the expression for \( y \) is:
    \[\boxed{y = \frac{(z - d)^2}{a^2}}\]
    i generally consider this the most concise and general expression for $y$.\\
    im aware you could expand \( (z - d)^2 \):
    \[y = \frac{z^2 - 2zd + d^2}{a^2}\]
    and additionally take into account of both possible factorisations:
    \[y = \frac{z^2+d(d-2z)}{a^2}\]
    \[y = \frac{d^2+z(z-2d)}{a^2}\]    
    
    \newpage
    \begin{tcolorbox}[title=\color{black}{\section{Q7}}, colback=white, colframe=black!30!white, boxrule=0.4mm, width=1\textwidth]
        Find a vector that is perpendicular to both of the vectors $$ \boldsymbol{a}=4 \boldsymbol{i}+3 \boldsymbol{j}+5 \boldsymbol{k} $$$$ \boldsymbol{b}=3 \boldsymbol{i}+4 \boldsymbol{j}-6 \boldsymbol{k} $$ Hence find a unit vector that is perpendicular to both \( \boldsymbol{a} \) and \( \boldsymbol{b} \).
    \end{tcolorbox}
    
    The unit vector is
    \[\mathbf{\hat{r}} =\frac{\vec{r}}{||{\vec{r}}||}\]
    Here we want to define $\vec{r}$ as a vector perpendicular to both $\mathbf{a}$ and $\mathbf{b}$, to do that we take there cross product \(\mathbf{a} \times \mathbf{b}\)
    \[\vec{r}=\mathbf{a} \times \mathbf{b} = \begin{vmatrix} \mathbf{i} & \mathbf{j} & \mathbf{k} \\ 4 & 3 & 5 \\ 3 & 4 & -6 \end{vmatrix}= -38\mathbf{i} + 39\mathbf{j} + 7\mathbf{k}\]
    Find the magnitude:
    \[||{\vec{r}}||=\|\mathbf{a} \times \mathbf{b}\| = \sqrt{(-38)^2 + 39^2 + 7^2} = \sqrt{3014}\]
    The unit vector is:
    \[\boxed{\frac{-38\mathbf{i} + 39\mathbf{j} + 7\mathbf{k}}{\sqrt{3014}}}\]

    \newpage
    \begin{tcolorbox}[title=\color{black}{\section{Q8}}, colback=white, colframe=black!30!white, boxrule=0.4mm, width=1\textwidth]
        If \( M=\left(\begin{array}{cc}7 & 9 \\ 1 & -2\end{array}\right) \) and \( N=\left(\begin{array}{cc}2 & 1 \\ -2 & 6\end{array}\right) \) find \( M N \) and \( N M \)
    \end{tcolorbox}
    
    Calculate \( MN \):
    \[MN = \begin{bmatrix} 7 & 9 \\ 1 & -2 \end{bmatrix} \begin{bmatrix} 2 & 1 \\ -2 & 6 \end{bmatrix}\]
    \[= \begin{bmatrix} 7 \cdot 2 + 9 \cdot (-2) & 7 \cdot 1 + 9 \cdot 6 \\ 1 \cdot 2 + (-2) \cdot (-2) & 1 \cdot 1 + (-2) \cdot 6 \end{bmatrix}\]
    \[\boxed{= \begin{bmatrix} -4 & 61 \\ 6 & -11 \end{bmatrix}}\]
    Calculate \( NM \):
    \[NM = \begin{bmatrix} 2 & 1 \\ -2 & 6 \end{bmatrix} \begin{bmatrix} 7 & 9 \\ 1 & -2 \end{bmatrix}\]
    \[= \begin{bmatrix} 2 \cdot 7 + 1 \cdot 1 & 2 \cdot 9 + 1 \cdot (-2) \\ -2 \cdot 7 + 6 \cdot 1 & -2 \cdot 9 + 6 \cdot (-2) \end{bmatrix}\]
    \[\boxed{= \begin{bmatrix} 15 & 16 \\ -8 & -30 \end{bmatrix}}\]
    
    \newpage
    \begin{tcolorbox}[title=\color{black}{\section{Q9}}, colback=white, colframe=black!30!white, boxrule=0.4mm, width=1\textwidth]
        If \( y=x^{4}-4 x^{3}-90 x^{2} \), find the values of \( x \) for which \( y^{\prime \prime}=0 \)
    \end{tcolorbox}
    
    First, find the first derivative:
    \[y' = \frac{d}{dx}(x^4 - 4x^3 - 90x^2) = 4x^3 - 12x^2 - 180x\]
    Find the second derivative:
    \[y'' = \frac{d}{dx}(4x^3 - 12x^2 - 180x) = 12x^2 - 24x - 180\]
    Set \( y'' = 0 \):
    \[12x^2 - 24x - 180 = 0\]
    Divide by 12:
    \[x^2 - 2x - 15 = 0\]
    Factor:
    \[(x - 5)(x + 3) = 0\]
    \[\boxed{x = 5,\ x = -3}\]
    
    \newpage
    \begin{tcolorbox}[title=\color{black}{\section{Q10}}, colback=white, colframe=black!30!white, boxrule=0.4mm, width=1\textwidth]
        Transpose \( b=g+t(a-3) \) to make \( a \) the subject
    \end{tcolorbox}
    
    Starting with:
    \[b = g + t(a - 3)\]
    this is easily rearranged to isolate \( a \):
    \[b - g = t(a - 3)\]
    \[\frac{b - g}{t} = a - 3\]
    \[\boxed{a = \frac{b - g}{t} + 3}\]
    
    \newpage\raggedright
    \thispagestyle{empty}
    \newgeometry{margin=1in,left=0.6in,right=0.6in,bottom=0in}
    
    \chapter{Part B: Computing}

    \hrule
    \vspace{3em}
    \begin{center}
        \Large\textbf{Notes:}
    \end{center}
    \vspace{1em}
    \begin{itemize} 
        \item All graphs and images were created using TikZ PGF. 
        \item Code implementations are original and written in MATLAB and to strictly follow guidelines with some python also.
        \textcolor{red}{In order to copy the code from my Github please follow the relevant links provided.} 
        \begin{itemize} 
            \item Honestly, relying only on MATLAB feels a bit short-sighted. its a temporary student license, and no point getting used to it. focusing solely on MATLAB in this course means missing out on essential programming skills, assets and applicability. While I \textbf{have to use MATLAB} for this module, it's not fully aligned with the long-term goals that it states.
            \end{itemize} 
            \item Please consider the answers as a whole. 
        \end{itemize}
   
    \newpage
    \setcounter{page}{11}
    \centering
    \fancyhead[L]{Part B: Computing}
    \newpage
    \newgeometry{margin=1in,left=0.6in,right=0.6in,bottom=0.4in}
   \begin{tcolorbox}[title=\color{black}{\section{Q1}}, colback=white, colframe=black!30!white, boxrule=0.4mm, width=1\textwidth]\centering
    The arc length of a segment of a parabola \(ABC\) of an ellipse with semi-minor axes \(a\) and \(b\) is given
    approximately by:
    \[L_{ABC} = \frac{1}{2}\sqrt{b^2+16a^2}+\frac{b^2}{8a}\ln\left(\frac{4a+\sqrt{b^2+16a^2}}{b}\right)\]
    \begin{tikzpicture}[scale=1.3,transform shape]
        \def\a{2} %main parameter
        \pgfmathsetmacro{\asqrt}{sqrt(\a)}
        \begin{scope}
            \clip (\asqrt+0.1, 0) rectangle (-\asqrt-0.1, \a+0.1);
            \draw[domain=-\asqrt:\asqrt, samples=100, smooth] plot (\x, {-(\x)^2 + \a});
        \end{scope}
        \draw[-latex] (-2.5, 0) -- (2.5, 0) node[right] {${x}$};
        \draw[-latex] (0, -0.1) -- (0, \a+0.5) node[above] {${y}$};%set reletive to 'a'can be changed
        \node[right] at (0, \a/2) {${a}$};
        \node at (-\asqrt-0.2, 0.25) {${A}$};
        \node at (\asqrt+0.2, 0.25) {${C}$};
        \node at (0.25,\a+0.2 ) {${B}$};
  
        \def\pad{0.3}
        \def\lin{0.2}
        \begin{scope}[shift={(0,-0.1)}]
             \draw[latex-] (-\asqrt, -0.2) -- (-\pad, -0.2);
            \draw[-latex] (\pad, -0.2) -- (\asqrt, -0.2);
            \node at (0,-0.2) {${b}$};
            \draw[-] (-\asqrt, -0.2+\lin) -- (-\asqrt, -0.2-\lin);
            \draw[-] (\asqrt, -0.2+\lin) -- (\asqrt, -0.2-\lin);
        \end{scope}
    \end{tikzpicture}\\[1em]
    Write a universal, user-friendly code, test your programme and determine $L_{ABC}$ if $a=11$ cm and $b=9$ cm.
    \end{tcolorbox}
    
    Since the goal is utilize Matlab, this question is fairly easy this version should suffice if a widely used and intuitive code is required.\\
    \vspace{2em}
    
    \lstinputlisting[style=matlab, title=\textbf{\textcolor{\link}{\texttt{\href{https://github.com/sakx7/mathcompuni/blob/main/matlab scripts/q1.m}{matlab scripts/q1\_easy.m}}}}]{matlab scripts/q1.m}
    \newpage
    
    
    
    \subsection{Advanced version}
    We are tasked with analyzing a parabola where:\\
    \vspace{1em}
    \hspace{3.8em}
    \begin{minipage}{0.6\textwidth}\centering
    \begin{itemize}[itemsep=-0.1cm]
        \item $a$ represents the height of the parabola
        \item $b$ represents the width of the parabola
    \end{itemize}
    \end{minipage}\\
    \vspace*{1em}
    My goal is to plot the parabola, first by formulating an equation in terms of these factors. Initially, we begin with a simple parabola in the general shape ($x^2$) similar to shown in the graph. Our goal is to select a reasonable function, with variables associated with the $x$ scale and $y$ location of the parabola:    
    \[y = -(\beta x)^2 + \gamma\]
    Given our problem statement:\\
    \vspace{1em}
    \hspace{3.8em}
    \begin{minipage}{0.8\textwidth}\centering
       \begin{itemize}[itemsep=-0.1cm]
        \item $\gamma$ represents the height, so $\gamma = a$
        \item The width is related to the roots of the equation when $y = 0$
    \end{itemize}
    \end{minipage}\\
    \vspace{1em}
    To find $\beta$, we solve:
    \[0 = -(\beta x)^2 + \gamma\]
    \[\beta = \frac{2\sqrt{\gamma}}{b}\]
    since we know $\gamma$:
    \[\beta = \frac{2\sqrt{a}}{b}\]
    Note: This requires $b \neq 0$ and $\sqrt{a} \neq 0$ (i.e., $a > 0$).\\
    Substituting $\gamma$ and $\beta$ into our original equation:
    \[\boxed{y = -\left(\frac{2\sqrt{a}}{b}x\right)^2 + a}\]
    This is our final parabola equation in terms of $a$ and $b$.
    The arc length of the parabola from $x = -\frac{b}{2}$ to $x = \frac{b}{2}$ is given by:
    \[L_{ABC} = \frac{1}{2}\sqrt{b^2 + 16a^2} + \frac{b^2}{8a}\ln\left(\frac{4a + \sqrt{b^2 + 16a^2}}{b}\right)\]
    When implementing this in a program:
    \begin{itemize}[itemsep=-0.1cm]
        \item Ensure that when there are no limit parameters for the inputs \(a\) and \(b\), the conditions \(a > 0\) and \(b \neq 0\) are appropriately handled.
        \item Update the plot of \(y = -\left(\frac{2\sqrt{a}}{b}x\right)^2 + a\) so that the sliders for \(a\) and \(b\) directly control the height and width, respectively.
        \item Calculate the arc length of the parabola using a function with inputs \(a\) and \(b\), and display the result in the plot.
    \end{itemize}
    \raggedright
    
    The rest is just nitpicky presentation, all just preference.
    
    \centering
    \lstinputlisting[style=mypythonstyle, title=\textbf{\textcolor{\link}{\texttt{\href{https://github.com/sakx7/mathcompuni/blob/main/py scripts/q1_advanced.py}{py scripts/q1\_advanced.py}}}}]{py scripts/q1_advanced.py}
    \includegraphics[scale=0.5]{figures/Figure_1.png}\\
    With the sliders, you can interactively update the plot and change $a$ and $b$ values in real time!
    

    \newpage
   \begin{tcolorbox}[title=\color{black}{\section{Q1}}, colback=white, colframe=black!30!white, boxrule=0.4mm, width=1\textwidth]\centering
   The voltage difference \(V_{ab}\) between points \(a\) and \(b\) in the Wheatstone bridge circuit is:
   \[V_{ab} = V \left(\frac{R_1 R_3 - R_2 R_4}{(R_1 + R_2)(R_3 + R_4)}\right)\]
   Write a universal, user-friendly program that calculates the voltage difference \(V_{ab}\).\\    
   Test your program using the following values:\\
   \vspace{1em}
    \begin{minipage}{0.4\textwidth}
    \centering
    \begin{itemize}[itemsep=-0.1cm]
        \item \(V = 14\) volts
        \item \(R_1 = 120.6\ \Omega\)
        \item \(R_2 = 119.3\ \Omega\)
        \item \(R_3 = 121.2\ \Omega\)
        \item \(R_4 = 118.8\ \Omega\)
    \end{itemize}
    \end{minipage}\hspace{-5em}
    \begin{minipage}{0.4\textwidth}
    \begin{circuitikz}[scale=1.3]
        \def\sc{1.3}
        \ctikzset{resistors/zigs=4, resistors/scale=.75,resistors/width=1.3}
        \draw
        node[ocirc] (A) at (2*\sc,0) {} 
        node[ocirc] (B) at (2*\sc,2*\sc) {}
        node[ocirc, label=left:$a$] (C) at (1*\sc,1*\sc) {}
        node[ocirc,label=right:$b$] (D) at (3*\sc,1*\sc) {}
        (A) to[short] (0.3,0) 
        to[american voltage source, v=$V$] (0.3,2*\sc)
        to[short] (B)
        (B) to[R, l_={$R_1$},] (C)
        to[R, l_={$R_2$}] (A)
        (B) to[R, l^={$R_3$}] (D)
        to[R, l^={$R_4$}] (A);
    \end{circuitikz}
\end{minipage}
    
    \end{tcolorbox}
    
    Using MATLAB is the aim, thus answering this issue is not too difficult. I don't see a necessity for an interactive or complicated presentation for this specific question. I could include user interface components that would let users enter numbers by clicking resistors or something similar, but given the limitations, I'll go with a more cautious strategy. It really just depends on how fascinating or even marginally intriguing I find the question.\\
    \vspace{2em}

    \lstinputlisting[style=matlab, title=\textbf{\textcolor{\link}{\texttt{\href{https://github.com/sakx7/mathcompuni/blob/main/matlab scripts/q2.m}{matlab scripts/q2.m}}}}]{matlab scripts/q2.m}
    
    \newpage

    

\end{document}
